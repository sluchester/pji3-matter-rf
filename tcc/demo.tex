%-------------------------------------------------%
% 2025-04-01 - Emerson Ribeiro de Mello - mello@ifsc.edu.br
%-------------------------------------------------%
% Sets aspect ratio to 16:9, and frame size to 160 mm by 90 mm.
\documentclass[aspectratio=169]{beamer}

% Sets aspect ratio to 4:3, and frame size to 128 mm by 96 mm
% \documentclass{beamer}

% Sets aspect ratio to 16:10, and frame size to 160 mm by 100 mm.
% \documentclass[aspectratio=161]{beamer}

\usepackage[english,brazil]{babel}


\usepackage[style=abnt,noslsn,justify]{biblatex}
\renewcommand*{\bibfont}{\small}
\usepackage{csquotes}
\addbibresource{referencias.bib}

% \pgfdeclareimage[width=3cm]{ifsclogo}{img/ifsc-logo-h-branco.png}


% Crie cores personalizadas para usar no tema
\definecolor[named]{novaCor}{HTML}{303A52}
\definecolor{verdeescuro}{RGB}{0,112,19}

% Outras cores já definidas no tema
% redwine
% azulcinza
% azulescuro
% cyanIFSC
% fundoIFSC
% cinzaescuro
% verdeifsc
% vermelhoifsc
% vermelhoifscclaro

% Theme parameters
\usetheme[
    % foreground=redwine, % cor do texto (usado no título dos slides) 
    % background=white, % cor do fundo dos slides
    % textfg=black, % cor do texto nos slides
    % titlefg=redwine, % cor do texto do slide de título
    % titlebg=white, % cor do fundo do slide de título
    % frametitlefg=white, % cor do texto do título do frame
    % frametitlebg=redwine, % cor do fundo do título do frame
    % sectionfg=redwine, % cor do texto do slide da seção
    % sectionbg=white, % cor do fundo do slide de seção
    % blocktitlefg=white, % cor do texto do título dos blocos
    % blocktitlebg=redwine, % cor do fundo do título dos blocos
    % blockbodyfg=cinzaescuro, % cor do texto dentro dos blocos
    % itemsep=7pt % Espaçamento entre os itens das listas 
]{ifsclean}



% -------------------------------------------------%
%              Título 
% -------------------------------------------------%
\title{Desenvolvimento de um Gateway de Integração para
Dispositivos RF Legados (Fixed Code e Rolling Code)
via Protocolo Matter}

\subtitle{Projeto de TCC 2 agora 2. Caraca deu bom memu de novo? }

\author{Luan de Barros Conceição sla}

%\institute{luanbarrossantos@gmail.com}

%\date{22 de dezembro de 2025}


%-------------------------------------------------%
\begin{document}
% -------------------------------------------------%

\begin{frame}[plain, noframenumbering]
    \maketitle
\end{frame}

\begin{frame}[plain, noframenumbering]{Licenciamento}
    \licenciamentoLivre
\end{frame}

\begin{frame}[plain, noframenumbering]{Agenda}
    \tableofcontents
\end{frame}
% -------------------------------------------------%

\section{Contexto e Motivação}

\begin{frame}{Contexto}
    \begin{itemize}
        \item Advento do eletromagnetismo
        \item Ondas eletromagnéticas
        \item Primeiras faixas de Frequência
        \item Necessidade de regulação
        \item ITU
        \item ANATEL
    \end{itemize}
\end{frame}

\begin{frame}{Motivação}
    \begin{itemize}
        \item A automação residencial cresce com foco em conforto, eficiência e controle remoto
        \item Grande parte dos dispositivos existentes utiliza RF em 433 MHz
        \item Operam de forma isolada, sem integração com assistentes modernos
        \item Fragmentação tecnológica
    \end{itemize}
\end{frame}


\begin{frame}{Problema de Pesquisa}
    \begin{itemize}
        \item Dispositivos RF legados (433 MHz) não se integram nativamente a ecossistemas IoT modernos
        \item Soluções atuais dependem de gateways proprietários e múltiplos aplicativos
        \item Falta de interoperabilidade entre tecnologias antigas e novos padrões de aplicação
    \end{itemize}
\end{frame}

\section{Objetivos}

\begin{frame}{Objetivos}
    \begin{itemize}
        \item Desenvolver um gateway e um aplicativo para integrar dispositivos RF 433 MHz a ecossistemas modernos via protocolo Matter
        \item Implementar captura e transmissão de sinais fixed code e rolling code
        \item Permitir a configuração manual do tipo de dispositivo via aplicativo
        \item Expor dispositivos como endpoints Matter
        \item Validar a integração com assistentes virtuais comerciais
    \end{itemize}
\end{frame}


\section{Fundamentação}

\begin{frame}{Por quê 433 MHz?}
    \begin{itemize}
        \item Faixa ISM livre e amplamente regulamentada
        \item Baixo custo de implementação
        \item Maior alcance e melhor penetração em obstáculos que 2,4 GHz
        \item Baixo consumo energético
        \item Uso consolidado em portões, sensores, alarmes e controles remotos
    \end{itemize}
\end{frame}


\begin{frame}{Modulações}
    \begin{itemize}
        \item ASK e OOK
        \item OOK:
        \begin{itemize}
            \item Implementação simplificada
            \item Eficiênciência energética
            \item Sensível a ruído
        \end{itemize}
    \end{itemize}
\end{frame}


\begin{frame}{Fixed Code e Rolling Code}
    \begin{itemize}
        \item \textbf{Fixed Code}
        \begin{itemize}
            \item Código estático transmitido a cada acionamento
            \item Implementação simples e baixo custo
            \item Vulnerável a ataques de repetição
        \end{itemize}
        \item \textbf{Rolling Code}
        \begin{itemize}
            \item Código dinâmico baseado em sincronização
            \item Maior segurança contra clonagem
            \item Maior complexidade de implementação
        \end{itemize}
    \end{itemize}
\end{frame}


\begin{frame}{Protocolo Matter}
    \begin{itemize}
        \item Padrão interoperável
        \item Transporte/Aplicação
        \item Descoberta comissionável de dispositivos
        \item Segurança
        \item Praticidade
    \end{itemize}
\end{frame}


\begin{frame}{Interoperabilidade e Protocolo Matter}
    \begin{itemize}
        \item Sanar a fragmentação entre ecossistemas IoT
        \item Matter surge como padrão aberto e interoperável
        \item Baseado em IP, com segurança nativa
        \item Compatível com múltiplos fabricantes e assistentes
        \item Pilha esp-matter da Espressif
    \end{itemize}
\end{frame}


\section{Proposta de Solução}
\begin{frame}{Proposta de Solução}
    \begin{itemize}
        \item Desenvolvimento de um gateway físico associado a um aplicativo móvel
        \item O usuário define manualmente:
        \begin{itemize}
            \item Tipo de protocolo (fixed ou rolling code)
            \item Tipo de dispositivo (lâmpada, tomada, portão etc.)
            \item Modo de operação (escuta ou emissão)
        \end{itemize}
        \item O gateway expõe o dispositivo como um endpoint Matter
    \end{itemize}
\end{frame}

% -------------------------------------------------%
%      fazer um diagrama de blocos aqui            %
% -------------------------------------------------%
\begin{frame}{Arquitetura do Sistema}
    \begin{itemize}
        \item Dispositivos RF 433 MHz operam como dispositivos de borda
        \item Gateway atua como intermediário e tradutor de protocolos
        \item Comunicação com hubs via Matter
        \item Controle realizado por assistentes virtuais compatíveis
    \end{itemize}
\end{frame}


\begin{frame}{Hardware Utilizado}
    \begin{itemize}
        \item \textbf{ESP 32}
        \begin{itemize}
            \item Conectividade Wi-Fi e BLE
            \item Suporte à pilha esp-matter
            \item Capacidade de processamento adequada ao gateway
        \end{itemize}
        \item \textbf{CC1101}
        \begin{itemize}
            \item Transceptor sub-GHz configurável
            \item Suporte a modulações OOK/ASK
            \item Maior robustez e seletividade
            \item Aprovadamente testado
        \end{itemize}
    \end{itemize}
\end{frame}


\begin{frame}{Firmware e Ponte Matter}
    \begin{itemize}
        \item Firmware desenvolvido com ESP-IDF
        \item Integração com a pilha esp-matter
        \item Módulos principais:
        \begin{itemize}
            \item Captura e transmissão RF
            \item Mapeamento de comandos
            \item Exposição de endpoints Matter
        \end{itemize}
        \item Gateway atua como Matter Bridge
    \end{itemize}    
\end{frame}

\begin{frame}{Aplicativo de Configuração}
    \begin{itemize}
        \item Aplicativo responsável pela configuração do sistema
        \item Comunicação local com o gateway
        \item Funções principais:
        \begin{itemize}
            \item Cadastro de dispositivos RF
            \item Definição de protocolo e tipo funcional
            \item Associação a endpoints Matter
        \end{itemize}
        \item \textbf{Elimina a necessidade de múltiplos aplicativos proprietários}
    \end{itemize}
\end{frame}


\section{Validação Experimental}
\begin{frame}{Validação Experimental}
    \begin{itemize}
        \item Testes realizados com dispositivos RF reais
        \item Cenários:
        \begin{itemize}
            \item Dispositivos Rolling Code
            \item Dispositivos Fixed Code
        \end{itemize}
        \item Integração validada com hubs compatíveis com Matter
        \item Avaliação de estabilidade, latência e confiabilidade
    \end{itemize}
\end{frame}

\section{Obrigado!}


% -------------------------------------------------%
%               standard Mello                     %
% -------------------------------------------------%

% \section{Listas}


% \begin{frame}{Lista de itens}
%     \begin{enumerate}
%         \item Primeiro item
%         \begin{itemize}
%             \item \Myhref{https://emersonmello.me}{Aqui tem um exemplo de \textit{link} para um \textit{site} usando o comando \texttt{Myhref}}
%             \item Segundo item
%         \end{itemize}
%         \item Segundo item
%         \item Terceiro item 
%         \begin{itemize}
%             \item Neste linha é apresentado um item que tem um texto grande que deverá ocupar mais de uma linha no slide. Assim, espera-se mostrar como será o alinhamento na margem esquerda
%             \item Segundo item
%         \end{itemize}
%     \end{enumerate}
% \end{frame}

% \subsection{Animações}

% \begin{frame}{Animação com lista de itens}
% \begin{itemize}[<+->]
%     \item\only<1>{Eu não sei quem criou o~\TeX}\only<2->{\st{Eu não sei quem criou o~\TeX}} 
%     \item O~\TeX~foi criado por Donald Knuth
%     \begin{itemize}
%         \item<.-> O~\TeX~é um sistema de tipografia 
%     \end{itemize}
%     \item Leslie Lamport criou o~\LaTeX~\cite{lamport94}
%     \begin{itemize}
%         \item<.-> O~\LaTeX~é um conjunto de macros para o~\TeX 
%     \end{itemize}
%     \item Em \textcite{knuth74} temos um dos trabalhos mais importantes de Donald Knuth
    
% \end{itemize}
% \end{frame}

% \begin{frame}{Uso do \texttt{alert}}
%     \begin{itemize}
%         \item<1-| alert@1> Aparecerá desde do slide 1 e receberá destaque no slide 1
%         \item<1-| alert@2> Aparecerá desde do slide 1 e receberá destaque no slide 2
%         \item<1-| alert@3> Aparecerá desde do slide 1 e receberá destaque no slide 3
%         \item<4-| alert@4> Aparecerá desde do slide 4 e receberá destaque no slide 4
%     \end{itemize} 
% \end{frame}

% \subsection{Colunas}

% \begin{frame}{Listas em colunas}
%     \begin{columns}[t, onlytextwidth]
%         \column{0.5\textwidth}
%             \begin{itemize}
%                 \item Item 1
%                 \begin{itemize}
%                     \item Subitem 1.1
%                     \item Subitem 1.2
%                 \end{itemize}
%                 \item Item 2
%                 \item Item 3
%             \end{itemize}
        
%         \column{0.5\textwidth}
%             \begin{enumerate}
%                 \item First
%                 \item Second
%                 \begin{enumerate}
%                     \item Sub-first
%                     \item Sub-second
%                 \end{enumerate}
%                 \item Third
%             \end{enumerate}
%     \end{columns}
% \end{frame}

% \begin{frame}{Listas e figuras}
% \begin{columns}
%     \column{.5\linewidth}
%      \begin{center}
%         \includegraphics[width=\linewidth]{figs/4x3.png}
%      \end{center}
%     \column{.5\linewidth}
%     \begin{itemize}
%         \item Item 1
%         \begin{itemize}
%             \item Subitem 1.1
%             \item Subitem 1.2
%         \end{itemize}
%         \item Item 2
%         \item Item 3
%     \end{itemize}
% \end{columns} 
% \end{frame}


% \section{Blocos}


% \begin{frame}{Blocos}
%     \begin{block}{Esse é um bloco}
%         Isso é um teste
%     \end{block}
%     \begin{block}{}
%     Bloco sem título	
%     \end{block}
%     \begin{alertblock}{Alerta}
%         Esse é um alerta
%     \end{alertblock}

%     \begin{exampleblock}{Exemplo}
%         Esse é um exemplo
%     \end{exampleblock}
% \end{frame}


% \begin{frame}{Blocos personalizados}

%     \begin{informacao}
%     Isso é uma mensagem de informação
%     \end{informacao}
    
%     \begin{informacaoazul}
%         Isso é uma mensagem de informação
%     \end{informacaoazul}

%     \begin{atencao}
%         Isso é uma mensagem de atenção
%     \end{atencao}
    
%     \begin{cuidado}
%         Isso é uma mensagem de cuidado
%     \end{cuidado}
% \end{frame}

% \begin{frame}[allowframebreaks]{Blocos personalizados}{Inspirados nos blocos Markdown do GitHub}
% \begin{nota}
%     Para apresentar uma informação útil
% \end{nota}

% \begin{tip}
%     Para apresentar uma dica
% \end{tip}

% \begin{important}
%     Informação importante que precisa ser destacada
% \end{important}

% \begin{warning}
%     Algo que precisa ser observado com atenção
% \end{warning}

% \begin{caution}
%     Para chamar a atenção do leitor para algo que pode ser perigoso
% \end{caution}
% \end{frame}


% \begin{frame}{Horários}
%     {Veja outros ícones em \url{https://ctan.org/pkg/fontawesome}}
%     \begin{itemize}
%         \item Horário das aulas 
%         \begin{caixa}[azul]{white}{\faCalendar}
%             \begin{itemize}
%                 \item 15:40 -- 17:30 - quarta-feira
%                 \item 13:30 -- 15:20 - quinta-feira
%             \end{itemize}
%         \end{caixa}
%         \item Atendimento extraclasse
%         \begin{caixa}[azul]{white}{\faCalendar}
%             \begin{itemize}
%                 \item 10:00 -- 12:00 - sexta-feira
%             \end{itemize}
%         \end{caixa}
%     \end{itemize}
% \end{frame}


% \section{Figuras}

% \begin{frame}{Figura 4x3}
%     \begin{center}
%         \includegraphics[height=.8\paperheight]{figs/4x3.png}
%     \end{center}
% \end{frame}
    

% \begin{frame}{Figura 16x9}
% \begin{center}
%     \includegraphics[width=.95\linewidth]{figs/16x9.png}
% \end{center}
% \end{frame}

% \section{Tabelas}

% \begin{frame}{Tabelas}
% \begin{center}
%     \begin{tabular}{l r r}\\\toprule
%         Produto & Quantidade & Valor (R\$) \\\midrule
%         Água    &      100   &   1,50 \\ 
%         Banana  &       10   &   3,00 \\
%         Maça    &        1   &   2,50 \\ \bottomrule
%     \end{tabular}
% \end{center}
% \end{frame}

% \section{Incluindo código fonte}


% \begin{frame}[fragile]{Olá mundo em C}
%     \lstinputlisting[style=ansic,caption={Olá mundo},label={cod:olac}]{codigos/olamundo.c}
% \end{frame}

% \begin{frame}[fragile]{Olá mundo em Java}
% \begin{lstlisting}[style=java]
% public class App{

%     public static void main(String[] args){
%         System.out.println("Olá mundo");
%     }

% }
% \end{lstlisting}
% \end{frame}

% Para permitir que o código fonte seja quebrado em mais de um slide. Se as referências couberem em um único slide, então pode omitir o uso do allowframebreaks
% \begin{frame}[allowframebreaks]{Referências}
%     % para incluir todas as referências do arquivo referencias.bib, mesmo sem citá-las explicitamente com os comandos \cite ou \textcite
%     \nocite{*}
%     \printbibliography
% \end{frame}
    
% Se não quiser usar o pacote biblatex, use o seguinte código
% \begin{frame}{Referências}
%     \nocite{*}
%     \bibliography{referencias}
%     \bibliographystyle{plain}
% \end{frame}

% \appendix

% \section{Slides de backup}

% \begin{frame}{Slide de backup}
%     \usebeamercolor[fg]{normal text}
%     Slides de backup são úteis para incluir materiais adicionais, necessários somente para ajudar a responder possíveis perguntas da plateia
%     \vfill
%     O pacote \texttt{appendixnumberbeamer} é usado para não numerar os slides de backup
% \end{frame}


\end{document}